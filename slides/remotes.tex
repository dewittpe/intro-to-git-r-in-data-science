\section{Remotes}
\begin{frame}
  \frametitle{Working with Remotes}
  \begin{itemize}
    \item So far we have only looked at working with git locally
    \item git is \emph{distributed} version control
    \item remotes are copies of the repository on
      \begin{itemize}
        \item another directory on your computer
        \item a network drive
        \item another computer
        \item a repository hosting server
          \begin{itemize}
            \item gitlab.com
            \item github.com
            \item bitbucket.org
            \item your own institutional server
          \end{itemize}
      \end{itemize}
  \end{itemize}
\end{frame}

\begin{frame}
  \frametitle{Working with remotes}
  \framesubtitle{Pros and Cons}
  \begin{itemize}
    \item Pros
      \begin{itemize}
        \item Collaboration!
        \item Every copy of the repo is a whole copy of the history - every
          copy is a back up for every other copy
        \item Additional tools can be used, depending on the remote/server
        \item Controls for who can read/write to the project
      \end{itemize}
    \item Cons
      \begin{itemize}
        \item Be very careful about publicly hosted repository hosts.  \emph{DO
          NOT COMMIT SENSITIVE DATA TO A REPO}
      \end{itemize}
  \end{itemize}
\end{frame}

\begin{frame}
  \frametitle{Setting Up a Remote}
  \begin{itemize}
    \item Let's create a github repository for the example project
    \item Introduce more git verbs
      \begin{itemize}
        \item push
        \item fetch
        \item merge
        \item pull
      \end{itemize}
  \end{itemize}
\end{frame}
