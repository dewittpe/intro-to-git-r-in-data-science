\section{Acquiring git}
\begin{frame}
  \frametitle{Git is Free!}

  \begin{itemize}
    \item Download from: \url{https://git-scm.com/download}

    \item Other options:

    \begin{itemize}
      \item Linux: install via your favorite package manager
        \begin{itemize}
          \item apt-get install git
          \item yum install git
        \end{itemize}
      \item Mac: Xcode command line tools
      \item Install from source: \url{https://github.com/git/git}
    \end{itemize}
  \end{itemize}
\end{frame}

\begin{frame}
  \frametitle{Terminal or GUI}
  \begin{itemize}
    \item git was built for the terminal
    \item Graphical User Interface (GUI) for git exists
      \begin{itemize}
        \item Gitkraken
        \item RStudio
        \item A list of GUI clients is available at \url{https://git-scm.com/downloads/guis/}
      \end{itemize}
  \end{itemize}
\end{frame}

\begin{frame}
  \frametitle{Set Up}

  \begin{itemize}
    \item You only need to do this once per machine:
    \item[]
    \item[] git config --global user.name ``Firstname Lastname''
    \item[] git config --global user.email ``first.last@institution.xxx''
    \item[]
    \item Use the terminal (git bash shell even on Windows) or some GUIs will
      support this
  \end{itemize}

\end{frame}
